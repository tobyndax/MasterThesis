\chapter{Conclusion and future work}
\section{Conclusion}
In this thesis I have shown that a physics simulation based on impulses with
calculations performed on the GPU is possible. Certain performance optimizations
have been implemented but there are as always still more to have if more time is
spent on the problem. In terms of performance the solution can match of the shelf
solutions (in the form of Bullet 2.86), when an object is decomposed into fewer than
218 particles.

\section{Future work}
Investigating the use of Shock Propagation as an alternative to the current scheme
were the velocity and impulse steps are reiterated.

Other methods of resolving the interpenetration could be a point of future work.

Making use of non-uniform structures such as oct-trees or kd-trees for the collision
detection, which would make the structure scale better with objects spread across
large distances.

Investigating the use of the Parallel-Sweep-And-Prune algorithm instead of the
uniform sorted grid for collision detection, a method which would most likely
benefit for larger objects as they work on bounding boxes and could be performed
in a two step fashion. First detecting whole bodies which may have collision.
Remove particles which belong to non-colliding bodies and redo the collision detection
on all remaining particles. In addition this method would support differently sized
particles better than the current one.

Investigate further optimizations to the velocity and impulse shader as they scale poorly.

he collision carrier structure, the contact manifold, currently only carries
4 collisions per particle. This limits the system in two ways, the system can miss
collisions when more than 4 collisions occur simultaneously on a particle and the
system is unfit to use differently sized particles. The larger particles would
be likely to experience more simultaneous collisions than four and the system would
lose needed information. Extending this collision transfer structure to become dynamic
in size would be an interesting and needed feature, however it would most likely have
to be extended in a clever way as not to impact performance too heavily.

Solid halfspheres, as described by~\cite{flex}, is the idea that the spheres that
constitute the bodies can have modified normals on the inside of the objects to counteract
tunneling. When two bodies become tangled the solid halfspheres modified normals
on the inside of the object ensure that the objects push apart and separate.
