\chapter{Introduction}\label{cha:intro}

\section{Purpose}

Synthetic data generation is a subject with increasingly high relevance today.
The data can be used for training and validation of self-learning systems,
something which is becoming increasingly more popular and useful within vision
based problems. Solving methods for high speed and high accuracy generation of these datasets
can greatly increase the productivity and development of these
types of algorithms.

For this thesis work I will investigate the use of physics
simulation for part distribution. For this thesis, several, identical models will
be 'poured' into a bin and simulated for their final distribution. In figure~\ref{fig:plb}, the situation
the thesis aims to simulate is visualized. Only the distribution of the parts
will be simulated, i.e. the arm will not be simulated. For these simulations
performance is of high interest as generation of large amounts of data generation
is one of the key aspects.

\begin{figure}[ht]
  \centering
  \includegraphics[width = 0.6\textwidth]{plb.png}
  \caption{The PLB system with parts in bin, which the thesis aims to reproduce.~\cite{fig:plb}}
  \label{fig:plb}
\end{figure}

\section{Limitations}
Since we are not interested in the exact final distribution for a certain start
state, only a valid final distribution is required, the correctness of the
simulation is an area where we can make some sacrifices. For instance a detailed
model for the friction might not grant a better final distribution than a simple one.
Additional limitation are hardware. The methods will be evaluated on the hardware
provided by SICK IVP. Graphics card: GeForce GTX 960 with 4 GB GDDR5 VRAM.
CPU: Intel Xeon W3530 at 2.8 GHz.
Memory: 6 GB RAM.

Initially the Rigid-body GPU pipeline of Bullet 3.x was to be investigated, however
it has become apparent that the implementation was more or less abandoned in 2013
when Erwin Coumans started working for Google instead of AMD. While operational,
no API for it exists. Due to it's state it is excluded from the thesis.

\section{Methodology}
Validation for physics simulations is a very difficult topic. Since the aim for
the thesis is to produce realistic enough distributions we do not need to focus
on absolutely realistic results but can instead evaluate the results in terms of
a few key properties. Of interest is: performance, since the more images
that can be generated the better the chances of finding rare problems; correctness,
as the simulations need to be relatively correct to give rise to the problems that
 happen in real life scenarios; concave collision, since the objects might
 be concave and contain holes or other cavities.

The methods will be evaluated in a comparative fashion against one another.

Two CPU methods, the rectangle bounding-box and HACD will be implemented using
BulletPhysics 2.83, an open-source, premissively licensed, optimized game physics
library. The third method will be a GPU method implemented from scratch using a
voxelized particle method with impulses.
