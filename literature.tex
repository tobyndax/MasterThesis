\chapter{Suggested literature}\label{cha:lit}
The following sources have been perused and investigated at the time of writing
and are intended to lay the foundation and present the initial plan of attack for
the problem.

\begin{itemize}

  \item flex, Matthias Muller et.al \newline
  A promising new technique from Nvidia. With a few exceptions the rigid body
  simulation uses the same approach as the one described in GPU Gems 3.

  \item GPU Gems 3, Chapter 29 Real-Time Rigid Body Simulation on GPUs \newline
  Voxelize the body into spheres. Solve the collisions as sphere-sphere collisions
  on the GPU. Binning the particles reduces the search time to O(n) instead of O($n^2$)

  \item Real-Time Rigid Body Interactions, Fredrik Fossum \newline
  An implementation heavily inspired by the GPU Gems 3 approach.

  \item Particle Simulation using CUDA, Simon Green \newline
  Some details on how to bin particles on the GPU.

  \item Polygons feel no pain, Ingemar Ragnemalm \newline
  For references to loads of useful 3D graphics core concepts.

  \item ...So How Can We Make Them Scream?, Ingemar Ragnemalm \newline
  Includes rigid body animation, Stencil buffers, GPGPU basics, quaternions and
  numerical integration techniques.

\end{itemize}

The following sources below are more leaned towards Bullet Physics. Bullet Physics'
experimental version 3.x include a GPU version of Rigid Body simulations which
might solve the problem very well. This would be interesting to investigate and
compare to the method(s) in the sources above.

\begin{itemize}

  \item GPU Rigid Body Simulation, Erwin Coumans \newline
  GDC presentation concerning BulletPhysics RigidBody GPU pipeline.

\end{itemize}
